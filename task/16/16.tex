% "Станет проще"

\documentclass[a4paper,12pt]{article} % тип документа

% report, book

%  Русский язык

\usepackage[T2A]{fontenc}			% кодировка
\usepackage[utf8]{inputenc}			% кодировка исходного текста
\usepackage{graphicx}
\usepackage[english,russian]{babel}	% локализация и переносы


%отступ
\usepackage[left=3cm,right=3cm,
    top=2cm,bottom=2cm,bindingoffset=0cm]{geometry}

% Математика
\usepackage{amsmath,amsfonts,amssymb,amsthm,mathtools} 
\usepackage{csvsimple}
\usepackage{multirow}

\usepackage{hyperref}
\usepackage{wasysym}
\usepackage{subcaption}
\usepackage{verbatim}
\usepackage{hyperref}
\usepackage{float}
\usepackage{enumerate}
\usepackage[dvipsnames]{xcolor}
\usepackage{rotating}

%Заговолок
%\graphicspath{ {images/} }


\begin{titlepage}
\author{Соловьянов Михаил }
\title{Задание 15. Электростатика.  Конденсаторы, работа поля.}
\date{\today}
\end{titlepage}



\begin{document} % начало документа
\maketitle
%http://easyfizika.ru/zadachi/elektrostatika/



\textit{Указание: Задачи оформлять ссылаясь на физические законы которые применяются для их решения.}




\section{ Конденсаторы }
\begin{enumerate}
	\item	Энергия $10^{-17} $ Дж, выраженная в эВ, составляет…
	%Источник: http://easyfizika.ru/zadachi/elektrostatika/energiya-10-17-dzh-vyrazhennaya-v-ev-sostavlyaet/\\
	\begin{turn}{180} 
	Ответ: $62.5 eV $ 
	\end{turn}


	\item	Заряд 5 нКл находится на расстоянии 0,45 м от поверхности шара диаметром 0,1 м, заряженного до потенциала 2400 В. Какую работу надо совершить, чтобы уменьшить расстояние между зарядов и шаром на 0,4 м?
	%Источник: http://easyfizika.ru/zadachi/elektrostatika/zaryad-5-nkl-nahoditsya-na-rasstoyanii-0-45-m-ot-poverhnosti-shara-diametrom-0-1-m/
	\begin{turn}{180} 
	Ответ: $4.8 \mu J  $
	\end{turn}

	\item В зазор между пластинами плоского конденсатора влетает электрон, пройдя перед этим ускоряющее поле с разностью потенциалов 25 кВ. Скорость электрона направлена параллельно пластинам конденсатора. Длина пластин 8 см, расстояние между ними 1 см. На сколько сместится электрон при выходе из зазора между пластинами, если разность потенциалов между ними 200 В?
	% Источник: http://easyfizika.ru/zadachi/elektrostatika/v-zazor-mezhdu-plastinami-ploskogo-kondensatora-vletaet-elektron-projdya-pered/
	\begin{turn}{180} 
	Ответ: $0.128 cm  $
	\end{turn}


	\item Три конденсатора электроемкостью 0,1, 0,125 и 0,5 мкФ соединены последовательно и подключены к источнику напряжения 800 В. Какая разность потенциалов будет на первом конденсаторе?
	%Источник: http://easyfizika.ru/zadachi/elektrostatika/tri-kondensatora-elektroemkostyu-0-1-0-125-i-0-5-mkf-soedineny-posledovatelno/
	\begin{turn}{180} 
	Ответ: $ 400 V  $
	\end{turn}
	
	\item Определить количество электрической энергии, перешедшей в тепло при соединении одноименно заряженных обкладок конденсаторов электроемкостью 2 и 0,5 мкФ, заряженных до напряжений 100 и 50 В, соответственно.
	%Источник: http://easyfizika.ru/zadachi/elektrostatika/opredelit-kolichestvo-elektricheskoj-energii-pereshedshej-v-teplo-pri-soedinenii-odnoimenno/
	\textit{Указание: Решать через энергии логично!}

\end{enumerate}


\section{Задачи части Б}

Из файла \textit{cap\_tasks.pdf} решить задачи 4,5,6,8,14. \\

Постарайтесь посидеть ,  подумать и даже если какие то задачи трудно даются на бумаге изобразить ход мыслей и какие то последовательности решений. Это часто дает балы.



\end{document}