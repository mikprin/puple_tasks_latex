% "Станет проще"

\documentclass[a4paper,12pt]{article} % тип документа

% report, book

%  Русский язык

\usepackage[T2A]{fontenc}			% кодировка
\usepackage[utf8]{inputenc}			% кодировка исходного текста
\usepackage{graphicx}
\usepackage[english,russian]{babel}	% локализация и переносы


%отступ
\usepackage[left=3cm,right=3cm,
    top=2cm,bottom=2cm,bindingoffset=0cm]{geometry}

% Математика
\usepackage{amsmath,amsfonts,amssymb,amsthm,mathtools} 
\usepackage{csvsimple}
\usepackage{multirow}

\usepackage{hyperref}
\usepackage{wasysym}
\usepackage{subcaption}
\usepackage{verbatim}
\usepackage{hyperref}
\usepackage{float}
\usepackage{enumerate}
\usepackage[dvipsnames]{xcolor}
\usepackage{rotating}

%Заговолок
%\graphicspath{ {images/} }


\begin{titlepage}
\author{Соловьянов Михаил }
\title{Задание 22. Повторение (электростатика).  Термодинамика.}
\date{\today}
\end{titlepage}



\begin{document} % начало документа
\maketitle

\section{Повторение}

Из Сборника Кравченко (\url{/lib/FTI_Kravchenko_Sbornik_fizika_1.pdf}) задачи:

\textbf{3.4 , 3.18 , 3.24 ,   3.27}


\section{термодинамика}

\subsection{}

Один моль разреженного гелия находится в горизонтальном цилиндрическом сосуде объемом $V$ с подвижным поршнем при температуре $Т$. Если при нагревании гелия объем увеличивается в 20 раз, а его теплоемкость Состается постоянной, то для определения приращения температуры $ \Delta T$ гелия, надо учесть следующие утвержде-ния ... 

\begin{enumerate}

\item теплоемкость  газа–величина,  равная  количеству  теплоты, необходимому для нагревания 1кг газа на 1К;
\item молярная  теплоемкость  газа  при  постоянном  объеме  равна приращению  внутренней  энергии  газа  в  количестве  1  моля  при повышении его температуры на 1К;
\item бесконечно  малое  количество  теплот,  сообщаемое  системе, расходуется  на  бесконечно  малое  приращение  ее  внутренней энергии и на совершение системой элементарной работы против внешних сил;
\item теплоемкость  газа–величина,  равная  количеству  теплоты, необходимому для нагревания газа на 1К
\item теплоемкость  газа  при  постоянном  объеме  равна  приращению внутренней энергии газа в количестве 1 моля при повышении его температуры на 1К

\end{enumerate}

\subsection{}
Один моль разреженного гелия находится в горизонтальном  цилиндрическом  сосуде  объемом $V = 10 $л  с  подвижным поршнем при температуре Т=300К. Если при нагревании гелия объем увеличивается в 20 раз, а его теплоемкость во всем процессе остается постоянной  и  равнойС=1000Дж/К,  то  приращение  температуры $ \Delta T$  гелия равно ... (Теплоемкостью сосуда и поршня пренебречь).



\end{document}