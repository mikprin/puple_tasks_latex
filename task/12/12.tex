% "Станет проще"

\documentclass[a4paper,12pt]{article} % тип документа

% report, book

%  Русский язык

\usepackage[T2A]{fontenc}			% кодировка
\usepackage[utf8]{inputenc}			% кодировка исходного текста
\usepackage{graphicx}
\usepackage[english,russian]{babel}	% локализация и переносы


%отступ
\usepackage[left=3cm,right=3cm,
    top=2cm,bottom=2cm,bindingoffset=0cm]{geometry}

% Математика
\usepackage{amsmath,amsfonts,amssymb,amsthm,mathtools} 
\usepackage{csvsimple}
\usepackage{multirow}

\usepackage{hyperref}
\usepackage{wasysym}
\usepackage{subcaption}
\usepackage{verbatim}
\usepackage{hyperref}
\usepackage{float}
\usepackage{enumerate}
\usepackage[dvipsnames]{xcolor}
%Заговолок
%\graphicspath{ {images/} }


\begin{titlepage}
\author{Соловьянов Михаил }
\title{Задание 12. Электростатика.  Потенциал. Работа поля.}
\date{\today}
\end{titlepage}



\begin{document} % начало документа
\maketitle
%http://easyfizika.ru/zadachi/elektrostatika/



\textit{Указание: Задачи оформлять ссылаясь на физические законы которые применяются для их решения.}




\section{Потенциал}
\begin{enumerate}






	\item Указать размерность единицы потенциала электростатического поля.

	\item Определить напряженность электрического поля в точке, находящейся на расстоянии 20 см от поверхности заряженной проводящей сферы радиусом 10 см, если потенциал сферы равен 240 В.
	\textit{Указание: Считать потенциал на удалении от заряженной сферы как:}
	\begin{equation}
	\varphi = \frac{kQ}{r}
	\end{equation}
	\textit{подробнее в материалах внизу}

	\item Капля росы в виде шара получилась в результате слияния 216 одинаковых капелек тумана. Во сколько раз потенциал поля на поверхности капли росы больше потенциала на поверхности капельки тумана?

	\item Два металлических шара, радиусы которых 5 и 15 см, расположенные далеко друг от друга, заряжены противоположными по знаку зарядами 12 нКл и -40 нКл. Шары соединяют тонкой проволокой. Какой заряд пройдёт по проволоке?

	\item Заряженная частица, пройдя ускоряющую разность потенциалов 600 кВ, приобрела скорость 5400 км/с. Определить массу частицы, если её заряд равен 2e.
	\textit{Комментарий: Очень жизненная задача!}

	\item Какую разность потенциалов должен пройти первоначально покоящийся электрон, чтобы приобрести кинетическую энергию 150 эВ?
	\textit{Указание: гугл в помощь уточнить что такое электрон-вольт}


	\item Между двумя горизонтально расположенными пластинами, заряженными до 10 кВ, удерживается в равновесии пылинка массой 2·10-10 кг. Определить заряд пылинки, если расстояние между пластинами 5 см?

\end{enumerate}



\section{Материалы}

\paragraph{Заряженные сферы}: \url{http://znaemfiz.ru/files/138426/potencial_pola_ravnomerno_zarazhennoj_sfery.pdf}



\end{document}