% "Станет проще"

\documentclass[a4paper,12pt]{article} % тип документа

% report, book

%  Русский язык

\usepackage[T2A]{fontenc}			% кодировка
\usepackage[utf8]{inputenc}			% кодировка исходного текста
\usepackage{graphicx}
\usepackage[english,russian]{babel}	% локализация и переносы


%отступ
\usepackage[left=3cm,right=3cm,
    top=2cm,bottom=2cm,bindingoffset=0cm]{geometry}

% Математика
\usepackage{amsmath,amsfonts,amssymb,amsthm,mathtools} 
\usepackage{csvsimple}
\usepackage{multirow}

\usepackage{hyperref}
\usepackage{wasysym}
\usepackage{subcaption}
\usepackage{verbatim}
\usepackage{hyperref}
\usepackage{float}
\usepackage{enumerate}
\usepackage[dvipsnames]{xcolor}
\usepackage{rotating}

%Заговолок
%\graphicspath{ {images/} }


\begin{titlepage}
\author{Соловьянов Михаил }
\title{Задание 15. Электростатика.  Конденсаторы, работа поля.}
\date{\today}
\end{titlepage}



\begin{document} % начало документа
\maketitle
%http://easyfizika.ru/zadachi/elektrostatika/



\textit{Указание: Задачи оформлять ссылаясь на физические законы которые применяются для их решения.}




\section{ Конденсаторы }
\begin{enumerate}
	\item	Два металлических шара радиусами 6 и 3 см соединены тонкой проволокой. Шары заряжены до потенциала 1500 В. Каково отношение заряда большего шара к заряду меньшего?\\
	\begin{turn}{180} 
	Ответ: 2 
	\end{turn}


	\item Шар радиусом 15 см, заряженный до потенциала 300 В, соединяют проволокой с незаряженным шаром. После соединения шаров их потенциал стал 100 В. Каков радиус второго шара?

	\begin{turn}{180} 
	Ответ: 0.3 m
	\end{turn}

	 \item Какой заряд пройдет по проводам, соединяющим пластины плоского воздушного конденсатора с источником тока напряжением 6,3 В, при погружении конденсатора в керосин? Площадь пластины конденсатора 180 см2, расстояние между пластинами 2 мм.

	\begin{turn}{180} 
	Ответ: 500 pC
	\end{turn}

	\ietm Конденсаторы электроемкостью 1 и 2 мкФ заряжены до разности потенциалов 20 и 50 В, соответственно. После зарядки конденсаторы соединены одноименными полюсами. Найти напряжение на этой батарее.

	\begin{turn}{180} 
	Ответ: 40 V
	\end{turn}
	%http://easyfizika.ru/zadachi/elektrostatika/kondensatory-elektroemkostyu-1-i-2-mkf-zaryazheny-do-raznosti-potentsialov-20-i-50-v/ 

\end{enumerate}


\section{Задачи части Б}

Из файла \textit{cap\_tasks.pdf} решить задачи 1,2,3. \\

Постарайтесь посидеть ,  подумать и даже если какие то задачи трудно даются на бумаге изобразить ход мыслей и какие то последовательности решений. Это часто дает балы.



\end{document}