% "Станет проще"

\documentclass[a4paper,12pt]{article} % тип документа

% report, book

%  Русский язык

\usepackage[T2A]{fontenc}			% кодировка
\usepackage[utf8]{inputenc}			% кодировка исходного текста
\usepackage{graphicx}
\usepackage[english,russian]{babel}	% локализация и переносы


%отступ
\usepackage[left=3cm,right=3cm,
    top=2cm,bottom=2cm,bindingoffset=0cm]{geometry}

% Математика
\usepackage{amsmath,amsfonts,amssymb,amsthm,mathtools} 
\usepackage{csvsimple}
\usepackage{multirow}

\usepackage{hyperref}
\usepackage{wasysym}
\usepackage{subcaption}
\usepackage{verbatim}
\usepackage{hyperref}
\usepackage{float}
\usepackage{enumerate}
\usepackage[dvipsnames]{xcolor}
\usepackage{rotating}

%Заговолок
%\graphicspath{ {images/} }


\begin{titlepage}
\author{Соловьянов Михаил }
\title{Задание 24. Работа и энергия.  Термодинамика.}
\date{\today}
\end{titlepage}



\begin{document} % начало документа
\maketitle

\section{Работа и энергия}

\subsection{Сбор дождевой воды}

Сперва посмотри видео: (\url{https://youtu.be/S6oNxckjEiE}). В нем изобретатель пытается использовать дожевую воду стекающую с крыши для выработки электричества.

Задача: Представим что ты тоже хочешь построить такую машину. Зная, что твой дом высотой $ 3 $м , а на крышу падает $ 0.2 \frac{L}{s} $ (литра в секунду) воды. Ты ставишь электрогенератор, который стоит у самого основания дома, и приводиться в действие водой стекающей в длинную трубу (как в видео, только потерями принебречь). Для идеальной ситуации (потерь в трении воды, и неидеальности генератора нет) найти:

\begin{enumerate}
	\item Мощность , которую вода сообщает во время дождя генератору?
	\item Какой ток будет выдавать зарядник твоего телефона подключенный к такому генератору, если КПД зарядника 80\%? (Телефоны как правило заряжаются от напряжения питания в 5В)
	\item За какое время удасться зарядить таким зарядником телефон с ёмкостью аккумулятора $ 4000 mAh $
	\item Какую работу совершит вода за 2 часа дождя?
\end{enumerate} 


\section{Термодинамика}

\subsection{}



\end{document}