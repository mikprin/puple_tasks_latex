% "Станет проще"

\documentclass[a4paper,12pt]{article} % тип документа

% report, book

%  Русский язык

\usepackage[T2A]{fontenc}			% кодировка
\usepackage[utf8]{inputenc}			% кодировка исходного текста
\usepackage{graphicx}
\usepackage[english,russian]{babel}	% локализация и переносы


%отступ
\usepackage[left=3cm,right=3cm,
    top=2cm,bottom=2cm,bindingoffset=0cm]{geometry}

% Математика
\usepackage{amsmath,amsfonts,amssymb,amsthm,mathtools} 
\usepackage{csvsimple}
\usepackage{multirow}

\usepackage{hyperref}
\usepackage{wasysym}
\usepackage{subcaption}
\usepackage{verbatim}
\usepackage{hyperref}
\usepackage{float}
\usepackage{enumerate}
\usepackage[dvipsnames]{xcolor}
%Заговолок
%\graphicspath{ {images/} }


\begin{titlepage}
\author{Соловьянов Михаил }
\title{Задание 11. Электростатика.  Начало. Заряд. Поле.}
\date{\today}
\end{titlepage}



\begin{document} % начало документа
\maketitle
%http://easyfizika.ru/zadachi/elektrostatika/



\textit{Указание: Задачи оформлять ссылаясь на физические законы которые применяются для их решения.}




\section{Закон кулона}
\begin{enumerate}

	\item В парафине на расстоянии 20 см помещены два точечных заряда. На каком расстоянии они должны находиться в воздухе, чтобы сила взаимодействия между ними осталась прежней?

	\item С какой силой ядро атома железа $ Fe^{56}_{26} $ притягивает электрон, находящийся на внутренней оболочке атома, расположенной на расстоянии $ 10^{-12} $ м?

	\item Два шарика, расположенные на расстоянии 10 см друг от друга, имеют одинаковые отрицательные заряды и взаимодействуют с силой $0,23 мН$ . Найти число избыточных электронов на каждом шарике.

	\item Точечные положительные заряды $q$ и $ 2q $ закреплены на расстоянии L друг от друга в вакууме. На середине прямой, соединяющей заряды, поместили точечный отрицательный заряд $−q$. Найти изменение модуля и направления силы, действующей на положительный заряд q?

	\item Вокруг отрицательного точечного заряда $-5$ нКл равномерно вращается по окружности под действием силы притяжения маленький заряженный шарик. Чему равно отношение заряда шарика к его массе, если шарик совершает 2 полных оборота в секунду, а радиус окружности 3 см?

	\item  Два одинаковых шарика, имеющих одинаковые заряды 1,6 мкКл, подвешены на одной высоте на нитях одной и той же длины. Расстояние между точками подвеса 0,2 м. Какой по величине и знаку заряд следует поместить на расстоянии 0,5 м от каждого из шариков, чтобы нити были параллельны?

\end{enumerate}


\section{Поля}
\begin{enumerate}






	\item Определить напряженность электрического поля, если на точечный заряд 1 мкКл действует кулоновская сила 1 мН.

	\item С какой силой действует однородное поле, напряженность которого 2 кВ/м, на электрический заряд 5 мкКл?

	\item Два точечных заряда 4 и -2 нКл находятся друг от друга на расстоянии 60 см. Определить напряженность поля в точке, лежащей посередине между зарядами.

	\item На какой угол отклонится бузиновый шарик с зарядом 4,9 нКл и массой 0,40 г, подвешенный на шелковой нити, если его поместить в горизонтальное однородное поле с напряженностью 100 кВ/м? \textit{6.2.31}

	\item  Точечный положительный заряд создаёт на расстоянии 10 см электрическое поле с напряженностью 1 В/м. Чему равна напряженность результирующего поля, если этот заряд внести в однородное электрическое поле с напряженностью 1 В/м, на расстоянии 10 см от заряда на линии, проходящей через заряд и перпендикулярной силовым линиям однородного поля.

	\item 3 заряда  велечиной $ Q $ помещены в вершины ромба со стороной $ L $ и  углом $ \alpha $, найдите результирующее поле в 4й вершине.

	\item 3 заряда велечиной $ Q $ расположены в вершинах правильного треугольника. Найдите силу которая действует на заряд $ -q  $ в центре этого треугольника. 
 


\end{enumerate}




\end{document}