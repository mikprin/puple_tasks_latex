% "Станет проще"

\documentclass[a4paper,12pt]{article} % тип документа

% report, book

%  Русский язык

\usepackage[T2A]{fontenc}			% кодировка
\usepackage[utf8]{inputenc}			% кодировка исходного текста
\usepackage{graphicx}
\usepackage[english,russian]{babel}	% локализация и переносы


%отступ
\usepackage[left=1cm,right=1cm,
    top=2cm,bottom=2cm,bindingoffset=0cm]{geometry}

% Математика
\usepackage{amsmath,amsfonts,amssymb,amsthm,mathtools} 
\usepackage{csvsimple}
\usepackage{multirow}

\usepackage{hyperref}
\usepackage{wasysym}
\usepackage{subcaption}
\usepackage{verbatim}
\usepackage{hyperref}
\usepackage{float}
\usepackage{enumerate}
\usepackage[dvipsnames]{xcolor}
\usepackage{rotating}
\usepackage{textcomp}

%Заговолок
%\graphicspath{ {images/} }


\begin{titlepage}
\author{Соловьянов Михаил }
\title{Задание 36. Задачи на Магнитное поле.}
\date{\today}
\end{titlepage}



\begin{document} % начало документа
\maketitle
%http://easyfizika.ru/zadachi/elektrostatika/

\section{Задачи попроще}
\subsection{}
Под каким углом расположен прямолинейный проводник с током 4 А в однородном магнитном поле с индукцией 15 Тл, если на каждые 10 см его длины действует сила 3 Н? Задача n8.1.1 из «Сборника задач для подготовки к вступительным экзаменам по физике УГНТУ»
%Источник: https://easyfizika.ru/zadachi/magnitnoe-pole/pod-kakim-uglom-raspolozhen-pryamolinejnyj-provodnik-s-tokom-4-a-v-odnorodnom-magnitnom/ 



\textit{Ответ: $\frac{\pi}{6}$}

\subsection{}
  Проводник с током 21 А и длиной 0,4 м перемещается в однородном магнитном поле с индукцией 1,2 Тл перпендикулярно к линиям индукции на расстояние 0,25 м. Какая при этом совершится работа? \\ \textit{ Задача n8.1.2 из «Сборника задач для подготовки к вступительным экзаменам по физике УГНТУ»}
%Источник: https://easyfizika.ru/zadachi/magnitnoe-pole/provodnik-s-tokom-21-a-i-dlinoj-0-4-m-peremeshhaetsya-v-odnorodnom-magnitnom-pole/ 
\textit{Ответ: $2520 mJ$}


\subsection{}

Проводник массой 5 г на метр длины, по которому течет ток силой в 10 А, расположенный перпендикулярно полю, оказался в состоянии невесомости. Какова индукция поля?\\ \textit{ Задача n8.1.11 из «Сборника задач для подготовки к вступительным экзаменам по физике УГНТУ»}
%Источник: https://easyfizika.ru/zadachi/magnitnoe-pole/provodnik-massoj-5-g-na-metr-dliny-po-kotoromu-techet-tok-siloj-v-10-a-raspolozhennyj/ 
\textit{Ответ: $5$ мТл}


\subsection{}

Электрон с энергией $4.2\dot 10^{-18}$  Дж влетает в однородное магнитное поле с индукцией 0,3 Тл перпендикулярно силовым линиям. Определить радиус траектории электрона. 
%Источник: https://easyfizika.ru/zadachi/magnitnoe-pole/elektron-s-energiej-4-2-10-18-dzh-vletaet-v-odnorodnoe-magnitnoe-pole-s-induktsiej-0-3-tl/ 

Ответ: 57,6 мкм

\subsection{}

Электрон, двигаясь равноускоренно из состояния покоя с ускорением 5 м/с2, в течение 1 мин приобрел скорость и влетел в однородное магнитное поле перпендикулярно линиям магнитной индукции. Определить индукцию поля, если сила Лоренца равна  $9.6\dot 10^{-17}$ Н.
%Источник: https://easyfizika.ru/zadachi/magnitnoe-pole/elektron-dvigayas-ravnouskorenno-iz-sostoyaniya-pokoya-s-uskoreniem-5-m-s2-v-techenie-1-min/ 


\subsection{}

Протон и дейтрон (ядро изотопа водорода $H^2_1 $) влетают в однородное магнитное поле перпендикулярно линиям индукции. Как связаны между собой периоды T1 и T2 обращения по окружностям, соответственно, протона и дейтрона (массы протона и нейтрона считать равными)? 
%Источник: https://easyfizika.ru/zadachi/magnitnoe-pole/proton-i-dejtron-yadro-izotopa-vodoroda-2h1-vletayut-v-odnorodnoe-magnitnoe-pole/ 
Ответ:  2 


\subsection{}

Если конденсатор с расстоянием между пластинами 1 см определенным образом расположить в однородном магнитном поле с индукцией 0,05 Тл, то ионы, летящие со скоростью 100 км/с, не испытывают отклонения. Найти напряжение на его обкладках. Вектор скорости перпендикулярен вектору магнитной индукции.
%Источник: https://easyfizika.ru/zadachi/magnitnoe-pole/esli-kondensator-s-rasstoyaniem-mezhdu-plastinami-1-sm-opredelennym-obrazom/ 




\end{document}